\section{Tropical Grassamanians}
\label{sec:tropical-grassmannians}
    We begin our exploration families of varieties by starting with one of the object that parametrizes subvarieties, the Grassmannians.
    The tropical Grassmannian arises from the ideal of quadratic Pl\"{u}cker relations 
    and it parametrizes the tropical linear spaces.  
    We will see that the lines in tropical projective space are trees, 
    and their tropical Grassmannian $\gr(2, n)$ is the sapce of the Speyer -Sturmfels phylogenetic tress that we will study in \citet{Billera2002},
    \citet{Speyer2004}. 
    
    Studying Grassmannians and Pl\"{u}cker's coordinates will first show us that
    under Pl\"{u}cker embedding, the Grassmannians are projective varieties,
    and further lead us to draw connections between the space of phylogenetic trees and moduli spaceso of curves. 
    References of this section are \citet{Maclagan2015}, 
    \citet{Speyer2003}, 
    \citet{Ranganathan2010},
    \citet{Hudec2007}
    
    
    First, we recall that the following definition.
    \begin{definition}[Tropical hypersurface]
    \label{def:tropical-hypersurface}
        Give a family of polynomials $F$ over an algebraically closed field $K$ with $n$ variables,
        the \textbf{tropical hypersurface} $T(F)$ of $F$ is the set of points in $\K^n$ such that the minimum is attained at least twice. 
        Equivalently, 
        $T(F)$ is the set of all points in $K^n$ where $F$ is not differentiable. 
    \end{definition}
    In the definition above, recall the Definition of tropical hypersurface in Section \ref{sec:convexity-polyhedral-complices-and-regular-triangulations}
    we mentioned that the tropical hypersurface is an $(n - 1)$-dimensional polyhedral complex in $K^n$. 
    
    %%%% Subsection
    \subsection{Grassmannians}
    \label{subsec:grassmannians}     
        \begin{definition}[Grassmannians]
        \label{def:grassmannians}
            Let $V$ be a vector space. 
            Grassmannian $\gr(r, V)$ is a space
            that parametrizes all the $r$-dimensional linear subspaces of the vector space $V$. 
            In another words 
            \[
            \gr(r, V) \coloneqq \{W \subset V: \dim W = r\}.
            \]
        \end{definition}
        
        \begin{example}[$\gr(2, 4)$]
        \label{ex:gr24}
            The set of all projective lines in the projective space $\P^3$ is $\gr(2, 4)$. 
        \end{example}
        
        An associative algebra defined on any vector space $V$ is called \textbf{exterior algebra}.
        The binary operation of any two element $v, w \in V$ is called \textbf{exterior product} or wedge product, denote as 
        \[
        v \wedge w
        \]
        such that $v \wedge w = - w \wedge v$. 
        \begin{example}
        \label{ex:v-wedge-v}
            For any $v \in V$, $v \wedge v = 0$.
            For any $v_i \in V$, 
            $i \in \lambda$ some arbitrary index set,
            $v_1 \wedge \cdots v_m = 0$ 
            whenever $v_i = v_{i + 1}$ for any $i \in \lambda$.
        \end{example}
        The wedge product of any two elements is called $2$-blade.
        Similarly,
        the wedge product of any $k$ elements is called
        $k$-blade. 
        
        \begin{definition}
        \label{def:exterior-power-of-a-vector-space}
            Let $V$ be a vector space.
            The $k$th exterior power of the vector space, denoted as $\bw^k V$ 
            is the span of all the $k$-blades of $V$.
            That is
            \[
            \bw^k V \coloneqq \vecspan{v_1 \wedge \cdots \wedge v_k: v_i \in V}. 
            \]
        \end{definition}
        
        \begin{definition}[Totally decomposible]
        \label{def:totally-decomposible}
            A element $w \in \bw^k V$ is said to be \textbf{totally decomposable}
            if it can be written as a $k$-blade;
            that is, if 
            \[
            w = w_1 \wedge \cdots \wedge w_k,
            \]
        \end{definition}
        
        \begin{example}[Non-example]
        \label{ex:not-totally-decomposable}
            The element $w = w_1 \wedge w_2 + w_3 \wedge w_4 \in \bw^k V$ is not totally decomposable. 
        \end{example}
        
        \begin{example}
        \label{ex:totally-decomposible-1}
            Every non-zero element of $\bw^1 V$ is totally decomposible. 
        \end{example}
        
        \begin{example}
        \label{ex:totally-decomposible-2}
            If $\dim V = 3$,
            then every non-zero element of $\bw^2 V$ is totally decomposable.
        \end{example}
        
        \begin{definition}[Divisor]
        \label{def:grassmannian-divisor}
            Let $w \in \bw^k V$,
            we say that $v \in V$ is a divisor of $w$ if there exists $u \in \bw^{k-1} V$ such that 
            \[
            w = v \wedge u. 
            \] 
        \end{definition}
        
        \begin{remark}
        \label{rem:space-of-divisors}
            One can see that if an element $w \in \bw^k V$ is totally decomposable,
            then the space of divisors is a subspace of dimension $k$ in $V$.
        \end{remark}
        
    %%%% Subsection
    \subsection{The Pl\"{u}cker Embedding of $\gr(r, V)$}
    \label{subsec:plucker-embedding-gr}
        \begin{lemma}
        \label{lem:plucker-embedding-lemma}
            Let $V$ be a vector space over some field $K$. 
            Let $W$ be a subspace of $V$ of finite dimension $r$ (i.e. $W$ is a point in $\gr(r, V)$).
            Let $\calb_1 = \vecspan{v_1, \ldots, v_r}$
            and $\calb_2 = \vecspan{w_1, \ldots, w_r}$.
            Then 
            \[
            v_1 \wedge \cdots \wedge v_r = \lambda(w_1 \wedge \cdots \wedge w_r)
            \]
            for some $\lambda \in K$. 
        \end{lemma}
        Note that $\lambda$ is the determinant of the change of basis matrix from $\calb_1$ to $\calb_2$.
        
        \begin{definition}[Pl\"{u}cker embedding]
        \label{def:plucker-embedding}
            A map $p: \gr(r, V) \rightarrow \P(\bw^r V)$ is defined as follows:
            given a subspace $W \in \gr(r, V)$ 
            and a basis $\calb_W = \vecspan{w_1, \ldots, w_r}$ of $W$,
            let $p: W \mapsto w_1 \wedge \cdots w_r$. 
            This map is the P\"{u}cker map
            and it embeds $\gr(r, V)$ into $\P(\bw^k V)$. 
        \end{definition}
        
        \begin{remark}
        \label{rem:plucker-embedding}
            Since different bases of $W$ will yield different multivectors of $\bw^r V$,
            but by Lemma \ref{lem:plucker-embedding-lemma},
            this map is unique up to scalar multiplication,
            hence a well-defined map on the \textbf{projectivization} of $\bw^r V$, 
            denoted as $\P(\bw^r V)$.
            We denote $p(W)$ as $[w]$.
            
            Notice that for any $v \in W$, 
            $p(W) \wedge v = 0 \in \bw^{k+1} V$.
        \end{remark}
        
        \begin{lemma}
        \label{lem:grassmannian-image-totally-decomposable}    
            $[w] \in \P(\bw^r V)$ 
            lies in the image of the Grassmannian 
            under the P\"{u}cker embedding 
            if and only if $w$ is totally decomposable.
        \end{lemma}
        The proof of this lemma is left to the readers.
        

        
        %%% Grassmannians as Projective Varieties
        \subsection{Grassmannians as projective varieties}
        \label{subsec:grassmannians-as-projective-varieties}
            We will use the Pl\"{u}cker embedding 
            and classify all the totally decomposable multivectors in $\bw^r V$ 
            as injections into the projective space $\P(\bw^r V)$. 
            
            \begin{theorem}[Injectivity of Pl\"{u}cker map]
            \label{thm:plucker-map-is-injective}
                The Pl\"{u}cker map is injective. 
            \end{theorem}
            
            \begin{proof}
                Define a map $\pi: \P(\bw^r V) \rightarrow \gr(r, V)$ as 
                \[
                \pi: [w] \mapsto \{v \in V| v \wedge w = 0 \in \bw^{r + 1} V\}.
                \]
                We want to show that $\pi \circ p = \id$;
                that is, $\pi \circ p(W) = W$ for any subspace $W \in \gr(r, V)$.
                Let $W \in \gr(r, V)$ with basis $\vecspan{w_1, \ldots, w_r}$.
                For any $w \in W$,
                $w \wedge [w_1 \wedge \cdots \wedge w_r] = 0$. 
                Thus $W \subseteq \pi \circ p(W)$.
                
                Let $v \in \pi \circ p(W)$, 
                then $v \wedge w_1 \wedge \cdots \wedge w_r = 0$,
                which implies $v$ is a linear combination of basis elements of $W$,
                and is thus in $W$. 
                
                Therefore, $W = \pi \circ p(W)$ 
                and $p$ is injective. 
            \end{proof}
            Such an injective map $p$ is indeed gives us the \textbf{Pl\"{u}cker embedding}.
            
            Now we identify the projectivization $\P(\bw^r V)$ with the projective space $\P^N$ as follows.
            For a basis $\calb_V = \vecspan{v_1, \ldots, v_n}$,
            we can choose $r$ basis elements to form a basis element of $\bw^r V$,
            each is linearly independent from others and the $\binom{n}{r}$ basis elements span the whole vector space $\bw^r V$ of dimension $\binom{n}{r}$.
            Projecting it, the projectivization $\P(\bw^r V)$ has dimension $N = \binom{n}{r} - 1$. 
            Thus we embed the Grassmannian $\gr(r, V)$ in $P(\bw^r V) = \P^n$ via the Pl\"{u}cker map. 
            That is 
            \[
            \gr(r, V) \subset \P^N, N = \binom{n}{r}.
            \]
            
            Now we define a similar map $\phi(w): V \rightarrow \bw^{d+1} V$ 
            for each $w \in \bw^r V$
            by $\phi_w(v) = w \wedge v$.
            Recalling Definition \ref{def:grassmannian-divisor}, 
            we notice that all the divisors of $w$ are in $\ker(\phi_w)$. 
            By Lemma \ref{lem:grassmannian-image-totally-decomposable}
            and Remark \ref{rem:space-of-divisors},
            we see that 
            \begin{corollary}
            \label{cor:rank-of-pi}
                For $w \in \bw^r V$, 
                $\phi_w$ has rank at most $n - r$.
                In particular, if $w$ is totally decomposable, $\phi_w$ has rank exactly $n-r$. 
            \end{corollary}
            
            Let us state the following theorem.
            \begin{theorem}
            \label{thm:grassmannian-is-projective-variety}
                The image of the Grassmannian is a projective variety.
                That is 
                \[
                p(\gr(r, V)) \subset P(\bw^r(V)).
                \]
            \end{theorem}
            This fact can be seen using the linear algebra fact that the rank of a matrix $M \in \mathsf{M}_{n\times m}(K)$ is the largest $r$ such that there exists non-zero $r \times r$ minors of $M$ and establishing a map $\psi: \bw^r(V) \rightarrow \hom(V, \bw^{d+1}(V))$ 
            such that $\psi(w) = \phi_w$.
            Using the maximality of the $(n - r) \times (n - r)$ minors, 
            a point $v \in \bw^r V$ lies in the locus of $(n - d + 1) \times (n - d + 1)$ minors of the matrix $\phi_w$, 
            which we view as a $n \times \binom{n}{d+1}$ matrix throughout this proof sketch.
            
            Another connection with linear algebra is the following.
            \begin{definition}[Pl\"{u}cker Coordinates]
            \label{def:plucker-coordinates}
                For a $r$-dimensional linear subspace $W \in \gr(r, V)$,
                the (homogeneous) coordinates of $p(W) \in \P^N$
                are the \textbf{Pl\"{u}cker coordinates} of $W$. 
            \end{definition}
            We remark that the coordinates are just 
            all the maximal non-vanishing minors of the matrix 
            whose rows are the basis of $W$. 
            
            \begin{example}
            \label{ex:plucker-coordinates-example-1}
                Consider the Pl\"{u}cker map of $\gr(1, n)$ maps a linear subspace $\vecspan{a_1 e_1 + \cdots + a_n e_n}$ is simply
                to the point $(a_1: \cdots : a_n) \in \P^{\binom{n}{1} - 1} = \P^{n-1}$. 
                Therefore, under this embedding, 
                $\gr(1, n) = \P^{n-1}$.  
            \end{example}
            
            \begin{example}
            \label{ex:plucker-coordinates-example-2}
                Consider the $2$-dimensional linear subspace 
                $W = \vecspan{e_1 + e_2, e_1 + e_3} \in \gr(2, 3)$.
                We can find its Pl\"{u}cker coordinates in $\P^{\binom{3}{2} - 1}$ in two ways.
                Take the wedge product of the basis elements, 
                $(e_1 + e_2) \wedge (e_1 + e_3) = 
                e_1 \wedge e_1 + e_2 \wedge e_1 + e_1 \wedge e_3 + e_2 \wedge e_3 = 0 - e_1 \wedge e_2 + e_1 \wedge e_3 + e_2 \wedge e_3$.
                Therefore, the coefficients this multivector is the Pl\"{u}cker coefficients of $W$ in $\P^2$;
                that is, $(-1, 1, 1)$. 
                
                Another way of seeing this is using the maximal minors othe matrix 
                \[
                M = 
                \begin{bmatrix}
                    1 & 1 & 0 \\
                    1 & 0 & 1 
                \end{bmatrix}
                \] where each row represents the coefficients of the canonical basis elements in the basis of $W$. 
                Calculating the maximal minors, we obtain the same answer $(-1, 1, 1)$.
                Notice that the any change of the spanning vectors only changes of the maximal minors by scalar multiplication,
                because the change is equivalanet to row operations on the matrix. 
                This example also confirms that the homogeneous Pl\"{u}cker coordinate of a point in $\gr(r, V)$ is well-defined. 
            \end{example}
            
            After working through Grassmannian and Pl\"{u}cker coordinates, 
            we will see how tropical Grassmannians are connected to phylogenetic trees in the next subsections. 
            
    \subsection{Space of Phylogenetic Trees}
    \label{subsec:grassmannians-and-space-of-phylogenetic-trees}
		A \textbf{phylogenetic tree} is tree $\tau$ 
		with finite labeled leaves
		and no vertices of degree two
		(because it must of degree three or above to avoid 
		bisecting an existing edge) 
		This is a notion arised from computational biology.
		We call the edges adjacent to the leaves of a phylogenetic 
		tree $\tau$ the \textbf{pendant edges} of $\tau$. 
		
		For any two leaves on a phylogenetic tree $\tau$,
		there exists a unique path 
		whose component edges have lengths.
		The sum of the lengths of all edges on a path between 
		leaves $i$ and $j$ is denoted  $d_{ij}$.
		The set of all distances, or metrics, on a tree $\tau$ 
		is called \textbf{tree metrics}
		and we thus obtain a metric space $[m]$. 
		
		For a tree $\tau$ with $n$ edges,
		we can put all the $\binom{n}{2}$ distances between 
		all pairs of $d_{ij}$ into a vector 
		$\mathbf{d} \in \R^{n \choose 2}$.
		Let $\Delta$ denote the set of all tree metrics in 
		$\R^{n \choose 2}$.
		The resultant metric space is called
		\textbf{space of phylogenetic trees}.
            
    %%% Subsection
    \subsection{Tropical Grassmannian and Tropical Linear Varieties from Phylogenetic Trees}
        In this subsection we study the tropicalization of Grassmannian
        and tropical linear varieties. 
        In particular, we will see how tropical linear spaces, 
        parametrized by Grassmannians, can correspond to phylogenetic trees.
        The recent work \citet{Dukkipati2013} proved that 
        there is a point on the tropical grassmannian that corresponds to each subtree of the phylogenetic tree.
        They also showed a necessary and sufficient condition for
        each of the subtree of a phylogenetic tree to be actually a fact of the tropical linear space. 
        The combinatorial structure of phylogenetic trees can reveal combinatorial structure of the tropicalization of the moduli spaces of curves. 
        
        The way we saw that $\gr(r, V)$ is a projective space in $\P^N$ 
        was through viewing $\gr(r, V)$ as the zero set of the maximal minors of a matrix. 
        Now we formally state that: 
        The Grassmannians are in fact the zero set of Pl\"{u}cker ideal,
        which is generated by all the Pl\"{u}cker relations.
        
        \begin{definition}
        \label{def:plucker-relation}
            For any two sequences $1 \le i_1 < \ldots < i_{r-1} < n$
            and $1 \le j_1 \le ldots \le j_{r+1} \le n$,
            the Pl\"{u}cker relation is defined as 
            \[
            \sum\limits_{m=1}^{r+1} (-1)^m p_{i_1, i_2, \ldots, i_{r-1}, j_m} p_{j_1, j_2,\ldots, \hat{j_a}, \ldots, j_{r+1}}
            \] where $\hat{j_a}$ means leaving out $j_a$ element.
        \end{definition}
        
        Let $I = \{i_l\}_{l=1}^{r-1}$, $J = \{j_k\}_{k=1}^{r+1}$.
        \begin{definition}[Pl\"{u}cker Ideal]
        \label{def:plucker-ideal}
            Let $I_{r, n}$ define the homogeneous ideal generated by all the Pl\"{u}cker relations, called \textbf{Pl\"{u}cker ideal}.
            That is,
            \[
            I_{r, n} = \inner{P_{I, J}: I, J \subseteq [n], |I| = r - 1, |J| = r + 1}
            \] where $P_{I, J}$ is the Pl\"{u}cker relations. 
        \end{definition}
        
        \begin{proposition}
        \label{prop:grassmannian-is-zero-setof-plucker-idea}
            The Grassmannian $\gr(r, V)$ is the subvariety of $\P^{\binom{n}{r} - 1}$ defined by this ideal.
        \end{proposition}
        
        \begin{definition}[Tropical Grassmannian]
        \label{def:tropical-grassmannian}
        The \textbf{tropical grassmannian} is $\trop(V(I_{r, n})$. 
        \end{definition}
        
        \begin{example}
        \label{ex:gr-24-plucker-ideal}
            For $\gr(2, 4)$, we can let $I = \{1\}, J = \{2, 3, 4\}$.
            \[
            P_{I, J} = -p_{12}p_{34} + p_{13}p_{24} - p_{14}p_{23}.
            \]
            Or equivalently, we could have $I = \{2\}, J = \{1, 3, 4\}$,
            or other $2$ possibilities, 
            which will give us the same relations up to sign.
            They all generate the same ideal, that is 
            \[
            I_{2, 4} = \inner{P_{\{1\}, \{234\}}} = \inner{P_{\{2\}, \{134\}}} = \inner{P_{\{3\}, \{124\}}} = \inner{P_{\{4\}, \{123\}}}.
            \]
        \end{example}
        
        \begin{example}
        \label{ex:gr-2n-plucker-ideal}
        In fact, the above example can be generalized to $\gr(2, n)$.
        We have
        \[
        I_{2, n} = \inner{p_{ij}p_{kl} - p_{ik}p_{jl} + p_{il} p_{jk}: i, j, k, l \in [n]}.
        \]
        The tropical Grassmannian 
        is thus $\trop(V(I_{2, n})) = \trop(V(p_{ij}p_{kl} - p_{ik}p_{jl} + p_{il} p_{jk}))$.
        Tropicalizing, we have that 
        \[
        \trop(V(I_{2, n})) = \{\text{the set of points where minimums of } p_{ij} + p_{kl},  p_{ik} + p_{jl},  p_{il} + p_{jk} \text{ is attained twice}\}
        \]
        \end{example}
        
        We state the following theorem from \citet{Dukkipati2013}.
        \begin{theorem}[Theorem 5.2, \citet{Dukkipati2013}]
            The Grassmannian $\gr(2, n)$ is equivalent to the space of all phylogenetic trees. 
        \end{theorem}
        For more detailed descriptions and study on the space of phylogenetic trees, we encourage readers to follow the refereces mentioned at the beginning of the section. 
        
        
        
        
        
        
            
            
            
            
            
            
            
            
            
            
        
        
        
        
        
        
        
        
        
        
        
        
        
        
        
    
    
    
    
    
    
    
    
    
    
    
    
    