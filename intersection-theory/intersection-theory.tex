\chapter{Intersection Theory}
\label{chp:intersection-theory}

\epigraph{
From time to time you rub so the liquid penetrates better, and otherwise you let time pass. The shell becomes more flexible through weeks and months -- when the time is ripe, hand pressure is enough, the shell opens like a perfectly ripened avocado!
}{Alexander Grothendick}

Intersection Theory stands in the center of algebraic geometry. 
It gives information about the intersection of subvarieties of a given variety. 
A very baby manifestation of the spirit of intersection theory is 
B\'{e}zout's Theorem

	\begin{theorem}[B\'{e}zout's Theorem]
		If two plane curves $A, B \subset \P^2$ intersect, 
		then they intersect at $(\deg A)(\deg B)$ points. 
	\end{theorem}
	
Another manifestation of intersection theory is Gauss' fundamental theorem of algebra:
	\begin{theorem}[Fundamental Theorem of Algebra]
		A non-zero, single-variable, 
		degree-$n$ polynomial with complex coefficients has, 
		counted with multiplicity, exactly 
		$n$ complex roots.
	\end{theorem}
We will introduce the notion of \emph{Chow ring} 
and its connection associated with homology and cohomology. 
Before we formally start and to give some flavor,
\textbf{Chow groups} are abelian groups associated to a geometric object
that are described as a group of cycles modulo an equivalence relation.
When the variety is smooth, 
the intersection product makes the Chow groups into a graded ring,
the Chow ring.
This Chow ring structure is parallel to the ring structure 
on the homology of a smooth compact manifold
that can be imported using Poincar\'{e} duality
from the natural ring structure on cohomology. 
Later we will see that 
the cohomology ring coincides in some cases with the Chow ring which facilitates our calculation of the cohomology ring. 

For readers who are unsatisfied with our very brief treatment of 
such a rich theory, 
we recommend the classic textbook \emph{Intersection Theory} 
by \citet{Fulton1998} for additional details.
Another textbook that treats this subject starting from an elementary level
is \emph{3264 and All That: A Second Course in Algebraic Geometry} by \citet{Eisenbud2016} and we are following their treatment in this manuscript as well. 



	







	











