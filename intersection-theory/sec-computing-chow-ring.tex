\section{Computing the Chow Ring}
\label{sec:computing-the-chow-ring}
To derive some techniques for computing the Chow ring,
we start with the idea of \textbf{affine stratification},
which will be a powerful tool in calculating the Chow groups 
for the projective spaces, Grassmannians and other rational varieties.
The idea is that given a variety (or a scheme) $X$ that admits a affine stratification,
we can decompose $X$ into union of affine spaces.
	\begin{definition}
		A scheme $X$ is \textbf{stratified} by a finite collection of 
		irreducible, locally closed subschemes $U_i$ 
		if $X$ is a disjoint union of the $U_i$ and,
		in addition,
		the closure of any $U_i$ is a union of $U_j$ 
		-- in other words, 
		if $\overline{U_i} \cap U_j \ne \varnothing$,
		then $U_j \subseteq \overline{U_i}$.
		
		The collection of such $U_i$ is called the \textbf{strata} 
		of the stratification. 
		The collections of such $\overline{U_i}$ is called 
		the \textbf{closed strata} of the stratification.	
	\end{definition}
In the same fashion, sometimes we call $U_i$ 
the open strata just to be clear. Now establish two more definitions based on stratification.
	\begin{definition}
	\label{def:stratification}
		We say that a stratification of $X$ with strata $U_i$ is
		an \textbf{affine stratification} 
		if each open stratum is isomorphic to 
		some affine $k$-space $\A^k$
		and we call it \textbf{quasi-affine} 
		if each $U_i$ is isomorphic to an open subset of some $\A^k$.
	\end{definition}

	\begin{example}[Strata of $\P$]
		The closed strata of $\P$ is just $\P$.
		However, the open strata of $\P$ is  
		$\P \backslash \P^{i - 1} \cong \A$
		(see Hartshorne for how affine spaces can be covered 
		by affine spaces).
	\end{example}
	
A generalization of the previous example is the following:
	\begin{example}[Strata of $\P^i$]
		The closed strata for $\P^i$ is still just $\P^i$,
		but the open strata are the affine spaces
		$U_i = \P^i \backslash \P^{i-1} \cong \A^i$. 
		Notice that this strata can be given by the 
		\textbf{flag} (a sequence of subspaces) 
		\[
		\P^0 \subset \P^1 \subset \cdots \subset \P^n.
		\]
	\end{example}

Now let us state a very important proposition
that will help us establish the techniques of using affine stratification 
to compute the Chow ring of a particular space
and make connections with the rational equivalence classes:
	\begin{proposition}
		If a scheme $X$ has a quasi-affine stratification,
		then $A(X)$ is generated by the classes of the closed strata.
	\end{proposition}