\section{Hyperplane Arrangement} 
\label{sec:hyperplane-arrangement}
	Let $\cala = \{H_i: 0 \le i \le n\}$ 
	be an arrangement 
	(a finite set with some geometric configuration) of $n+1$ 
	hyperplanes in $\P^d$
	such that no hyperplanes intersect 
	with each other at infinitely many points up to translation.
	We are interested in how the complement $X = \P^d \setminus \cala$ 	is configured by these hyperplanes. 
	In fact, $X$ is a subvariety of the torus $T^n$,
	cut out by a linear system of equations.

	Let $\mathbf{b}_i \in K^{d+1}$ be the normal vector of 
	the hyperplane $H_i$;
	that is $H_i = \{\mathbf{z} \in \P^d: \mathbf{b} \cdot \mathbf{z} = 0\}$.
	Let $B \in \mathsf{M}_{(d+1)\times (n+1)}$ 
	with columns being $b_i$ 
	and $A \in \mathsf{M}_{(n-d)\times (n+1)}$ 
	with rows being the basis for the kernel of $B$.
	Let $I$ be the ideal in $K[x_0^{\pm 1}, \ldots, x_n^{\pm 1}]$ 
	generated by the linear forms 
	\[
	f_i = \sum\limits_{j = 0}^{n} a_{ij} x_j
	\] for $1 \le i \le n-d$.
	
	We embed $X$ into $T^n$ as a subvariety 
	as the following proposition.
	
		\begin{proposition}
			There is an isomorphism $\phi: X \rightarrow T^n$ 
			between 
			the arrangement complement $X = \P^d \setminus \cala$ 
			and the subvariety $V(I)$ of $T^n$,
			defined as 
			\[
			\mathbf{z} \mapsto (\mathbf{b}_0 \cdot \mathbf{z}:
			\cdots: \mathbf{b}_n \cdot \mathbf{z}).
			\]
		\end{proposition}
	First notice that our assumption about the hyperplane 
	makes them span the whole space.
	Also notice that the image of this isomorphism 
	is indeed contained in the torus $T^n$,
	since $\mathbf{z} \ne \mathbf{0}$
	and that the ideal is fixed by the diagonal action by $K^\ast$
	(so we mod out the $K^\ast$). 
	We leave it to the readers to check that 
	this map is an isomorphism.
	
	\begin{example}[Three Lines in $\P^2$]
		Let $\cala$ be the hyperplane arrangement in $\P^2$
		that consists of the following four lines 
		with corresponding normal vectors
		\begin{align*}
			H_0 &= \{x_1 = x_2\}, \mathbf{b}_0 = \threevector{0}{1}{-1}, \\
			H_1 &= \{x_0 = x_2 \}, \mathbf{b}_1 = \threevector{1}{0}{-1},\\
			H_2 &= \{x_0 = x_1 \}, \mathbf{b}_1 = \threevector{1}{0}{-1}.
 		\end{align*}
		Thus we have a matrix $B \in \mathsf{M}_{3 \times 3}$:
		\[
		B = 
		\begin{bmatrix}
			0 &  1 & 0 \\
			1 & 0 & 0 \\
			-1 & -1 & -1 
		\end{bmatrix}
		\]
		and the matrix $A \in \mathsf{M}_{3 \times 3}$ 
		can be chosen to be:
		\[
		A = 
		\begin{bmatrix}
			1 &  0 & -1 \\
			0 & 1 & 1 \\
			1 & -1 & 0 
		\end{bmatrix}.
		\]
		Thus the ideal defined by $A$ is thus 
		\[
		I = \inner{x_1 - x_3, x_2 + x_3, x_1 - x_2} \subset k[x_1^{\pm 1}, \ldots, x_5^{\pm 1}].
		\]
		This linear ideal defines a plane in $\P^4$ 
		and $V(I)$ is the intersection of that plane with torus $T^4$.
		The previous proposition identifies the linear variety $V(I)$ 
		with the complement $\P^2\setminus \cala$ 
		of our arrangement of three lines in the plane. 
	\end{example}
	
	In fact any ideal generated by linear forms 
	corresponds to some hyperplane arrangement;
	even if the linear forms are not homogeneous, 
	we can homogenize it. 
	
	To illustrates the connection between hyperplane arrangements
	and matroid theory,
	we introduce the following definitions
	\begin{definition}[Support]
		The \textbf{support} of a linear form 
		$l = \sum a_i x_i \in I$ is 
		$\supp(l) = \{i : a_i \ne 0\}$.
	\end{definition}
	
	\begin{definition}[Circuit]
	\label{def:circuit}
		A non-empty subset $C$ in $[n]$ is a circuit of $I$ 
		if $C = \supp(l)$ for some non-zero linear form $l$ 
		in the ideal $I$
		and $C$ is inclusion-free.
	\end{definition}
	
	\begin{remark}
		The inclusion-free requirement makes a circuit $C$ 
		a maximal independent subset of the linear forms
		since it must be reduced to the set 
		such that $C$ does not contain or be contained 
		in some other circuit. 
		This implies that a circuit in $I$ is also 
		a minimal linear dependence 
		of the columns of $B$, the $\mathbf{b}_i$'s. 
		
		
		With this property, 
		we point out that the set of linear forms $l_C$ in $I$ 
		is the union of all reduced Gr\"{o}bner bases for $I \cap K[x_0, \ldots, x_n]$.
		Also each circuit is one-to-one correspond to a linear form 
		in $I$. 
		The ideal also has at most $\binom{n+1}{d+2}$ circuits, 
	 	which is attained when all the minors of $B$ are non-zero. 
		This implies that when computing Gr\"{o}bner bases,
		the problem is essentially reduced to 
		a Gaussian elimination problem. 
	\end{remark}
	
	Let $I \subset K[x_0^{\pm 1}, \ldots, x_n^{\pm 1}]$ 
	be generated by linear forms,
	where $K$ has the trivial valuation,
	and consider the hyperplane arrangement complement 
	that is identified with the variety of an ideal of linear forms, 
	$X \cong V(I)$.
	We restate the above remark in our tropical language
	
	\begin{proposition}
		The set of linear polynomials $l_C$ in $I$
		whose supports are circuits is a tropical basis for $I$.
	\end{proposition}
	
	To see more geometric intuition, we state another version of 
	this proposition.
	\begin{proposition}
		A vector $\mathbf{w} \in \R^{n+1}/\R$ lies in $\trop(X)$ 
		if and only if, for any circuit $C$ of the ideal $I$,
		the minimum of the coordinates $w_i$ is attained at least twice,
		as $i$ ranges over all circuits in $C$.
	\end{proposition}
	
	\begin{example}[Homogenizing Ideals of Linear Forms]
		Consider the ideal 
		\[
		I' = \inner{1 + x_1 + x_2, x_1 + 42x_2 + 47 x_3}
		\subseteq K[x_1^{\pm 1}, \ldots, x_3^{\pm 1}]
		\]
		which defines a two dimensional subvariety $X$ 
		of $T^3$ because we have two linear forms
		with three coordinates.
		Homogenizing the ideal $I'$,
		we obtain 
		\[
		I = \inner{x_0 + x_1 + x_2 + x_3, x_1 + 42x_2 + 47 x_3}
		\subseteq K[x_0^{\pm 1}, \ldots, x_3^{\pm 1}].
		\]
		Thus the variety $X = V(I)$ is the complement of 
		some hyperplane arrangement of $4$ lines in the projective plane $\P^2$. 
	\end{example}
	
	\begin{example}[A Tropical Basis]
		Let $I$ be the homogenized ideal in the previous example:
		\[
		I = \inner{x_0 + x_1 + x_2 + x_3, x_1 + 42x_2 + 47 x_3}
		\subseteq K[x_0^{\pm 1}, \ldots, x_3^{\pm 1}].
		\]
		The set of circuits is 
		\[
		C_1 = \{0, 1, 2, 3\}, C_2 = \{1, 2, 3\}
		\]
		but it is not inclusion-free at this moment;
		we reduce it to the set of circuits
		\begin{align*}
			C_1 &= \{1, 2, 3\} \\
			C_2 &= \{0, 1, 2\} \\
			C_3 &= \{0, 1, 3\} \\
			C_4 &= \{0, 2, 3\}.
		\end{align*}
		By ranging over all this set of circuits and attaining minimum twice, we can obtain the tropical basis. 
	\end{example}

	We call such $\trop(X)$, a \textbf{tropicalized linear variety}.
	Later we will clarify the difference between 
	tropical linear variety and tropicalized linear variety. 
	The notion of circuits give us a combinatorial description 
	of the tropicalization of $\trop(X)$ 
	for a linear vareity $X$
	using teh the Gr\"{o}bner basis.
	To organize all the information embedded in the circuits
	-- their representation of minimal linear dependence 
	among all the column vectors $\mathbf{b}_i \in K^{d+1}$ 
	of the matrix $B$.
	
	\begin{definition}[Lattice of Flats]
	\label{def:lattice-of-flats}
		Given a linear variety $X$, 
		\textbf{lattice of flats} $\call(B)$ 
		the set of subspaces (flats) of $K^{d+1}$ 
		that are spanned by subsets of the column vectors of $B$.
	\end{definition}
	
	Organizing the lattice of flats into a poset $\call(B)$ of rank $d+1$
	according to subspace relations with abuse of notation,
	we obtained a \textbf{simplicial complex},
	set composed of points, line segments, triangles, 
	and their $n$-dimensional counterparts,
	and we call it \textbf{order complex} of the poset.
	
