\section{Geometric Tropicalization}
\label{sec:geometric-tropicalization}
    Given a subvariety (or subscheme) $X \subset T^n$ embedded in a torus,
    we can see that the tropical variety $X^{\trop}$ in fact gives a good compactification of $X$
    using a technique called \textbf{geometric tropicalization}.
    Geometric tropicalization was first used to study compactification of subvarieties in \citet{Tevelev2007} and compactifications of moduli spaces of del Pezzo surfaces in \citet{Hacking2007}. References in this section are 
    \citet{Maclagan2015}, \citet{Payne2014},
    \citet{Cavalieri2014}
    and 
    \citet{Cueto2011}.
    
    \begin{definition}[Boundary, Divisorial]
    \label{def:boundary-divisorial}
        Let $X$ be a very affine variety,
        and $\overline{X}$ is a compactification of $X$.
        We call the set $\partial \overline{X} = \overline{X}/X$
        the \textbf{boundary}.
        If the boundayr $\partial \overline{X}$ is a union of codimension-one subvarieties of $\overline{Y}$, 
        we say the boundary is \textbf{divisorial}.
    \end{definition}
    
    \begin{definition}[Combinatorial normal crossings divisor]
    \label{def:combinatorial-normal-crossings-divisor}
        Let $D_1, \ldots, D_l$
        be the irreducible components of $\partial \overline{X}$.
        The boundary $\partial \overline{X}$ is a \textbf{combinatorial normal crossings divisor}
        if, 
        for any subset $\sigma \subseteq \{1, \ldots, l\}$,
        the intersection $\cap_{i \in \sigma} D_i$ has codimension $|\sigma|$ in $\overline{X}$.
    \end{definition}
    
    \begin{definition}[Simple Normal Crossings]
    \label{def:simple-normal-crossings}
        Let $\partial \overline{Y}$ be a combinatorial normal crossings divisor.
        for any subset $\sigma \subseteq \{1, \ldots, l\}$,
        the intersection $\cap_{i \in \sigma} D_i$ is transverse,
        then the boundary is \textbf{simple normal crossings}.
    \end{definition}
    Here we say that if an intersection is \textbf{transverse}, then the intersection will have codimension equal to the sums of the codimensions of the two interesecting $D_i, D_j$.
    That is, transversality characterizes the most generic intersections.

    Given a variety (or a scheme) $X$
    with a simple normal crossing boundary divisor $D$,
    there exists a map $\psi$ from $X \setminus D$ to a torus $T$,
    such that this map is an embedding.
    The map induces a map from the dual intersection complex $\Sigma$ to the vector space of one paramter subgroupos of $T$, which we briefly discussed in Section \ref{sec:toric-variety},
    thus giving a fan structure on $\Sigma$. 
    
    Following the discussion of \citet{Cueto2011},
    we see that the top dimensional cones of $\Sigma$ associated with a weight function in fact produces a balanced fan.
    This fan gives a toric variety,
    allowing us to consider the closure of image of the map $\psi$.
    That is, we consider $\overline{\psi(X \setminus D}$ and its relationship with $X$. 
    
    Geometric tropicalization is an important techniques 
    that we will be use to understand the combinatorial structure of $\calm_{0, n}^{\trop}$ and the boundary stratification of $\overline{\calm}_{0, n}$ (see Definition \ref{def:stratification}).
    We will use geometric tropicalization to understand compactification of moduli spaces of smooth $n$-pointed curves and weighted curves in Section \ref{sec:geometric-tropicalization-for-m-0n}.
    For an interested reader, 
    we apologize for not diving into the details of tropical compactification, but we encourage readers to look into work \citet{Tevelev2004} and \citet{Maclagan2015}. 

    
    
    
    
    
    
    