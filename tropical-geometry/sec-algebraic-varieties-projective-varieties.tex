\section{Algebraic Varieties and Projective Varieties} 
\label{sec:algebraic-varieties-and-projective-varieties}
   In very short words, classical algebraic geometry is a study of zero loci of polynomials. 
   Given a family of polynomials, 
   an algebraic variety is, despite its name, a geometric object 
   that describes all the points in a space that vanish on 
   all the polynomials in the family.
   Throughout the section, 
   we let $k$ be a fixed algebraically closed field.
   We are following the treatment of \citet{Hartshorne1977}.
   
   \begin{definition}[Affine $n$-space set]
	We define \textbf{affine $n$-space} over $k$,
	denoted $\A_n^k$ or simply $\A^n$,
	to be the set of all $n$-tuples of elements of $k$.
	An element $P \in \A^n$ will be called a \textbf{point},
	and if $P = \{a_1, \ldots, a_n\}$ with $a_i \in k_i$,
	then the $a_i$ will be called the \emph{coordinates} of $P$.
   \end{definition}
   
   For this section, we keep the following notations:
   Let $A = k[x_1, \ldots, x_n]$ be the polynomial ring in $n$ variables 
   over $k$.
   For a polynomial $f \in A$, we can define the zeros of $f$ as follows:
   \[
   Z(f) = \{P \in \A^n | f(P) = 0\}
   \] where $P = (a_1, \ldots, a_n), a_i \in k$.
   To extend the definition of ``zero set" when given a family 
   of polynomials $T \subseteq A$,
   we have 
   \[
   Z(T)  = \{P \in \A^n | f(P) = 0 \text{ for all } f \in T\}. 
   \]
   
   \begin{example}
   	If $f = 0 \in A = k[x]$, $Z(f) = \A$. 
   \end{example}
   
   \begin{example}
   	If $f = \alpha \in \C \setminus \{0\}$, $Z(f) = \varnothing$. 
   \end{example}
   
   \begin{example}
   	Let $\A = \C$, $A = k[x, y]$.
	Let $f \in A$ be $x^2 + y^2 - 1$.
	Then we can see that $Z(f)$ is the unit circle in the $\C^2$.
   \end{example}
   
   \begin{example}
   	Let $a$ be an ideal of $A$, then 
	\[
	Z(a) = \{P \in \A^n | f(P) = 0 \text{ for all } f \in T\}. 
	\]
   \end{example}
   
   \begin{definition}[Algebraic Set]
   	A subset $Y$ of $\A^n$ is an \textbf{algebraic set} 
	if there exists a subset $T \subseteq A$ such that 
	$Y = Z(T)$. 	
   \end{definition}
   
   Given two algebraic sets $Y_1, Y_2$ such that 
   $Y_1 = Z(T_1), Y_2 = Z(T_2)$ for some $T_1, T_2 \subseteq A$,
   then the $Y_1 \cup Y_2$ is precisely the points that vanish 
   polynomials either in $T_1$ or in $T_2$,
   which can be described mathematically as 
   $Y_1 \cup Y_2 = Z(T_1 T_2)$
   where $T_1 T_2 = \{fg | f \in T_1, g \in T_2\}$.
   Similarly, if $Y_\alpha = Z(T_\alpha)$ is the algebraic set of 
   any family of polynomials for some arbitrary index $\alpha$,
   then $\cap Y_\alpha = Z(\cup T_\alpha)$. 
   Every point in $\A^n$ vanish on empty set of polynomials,
   thus $\A^n = Z(\varnothing)$. 
   All the above can be summarized in the following proposition.
   
   \begin{proposition}
   	The union of two algebraic sets is an algebraic set. 
	The intersection of any family of algebraic sets is an algebraic set.
	The empty set and the whole space are algebraic sets.
   \end{proposition}
   
   The above notion of taking union and arbitrary intersection 
   resembles the axioms of a topology on a topological space. 
   \begin{definition}[Zariski Topology]
   	The \textbf{Zariski Topology} has the complements of the algebraic sets as its open sets. 
   \end{definition}
   We leave it to the readers to check that 
   the collection of complements of the algebraic sets is a topology 
   using the axioms. 
   The process should be similar to the above proposition. 
   
   \begin{definition}[Irreducible]
   	A nonempty subset $Y$ of a topological space $X$ is irreducible 
	if it cannot be expressed as the union of $Y = Y_1 \cup Y_2$
	of two proper subsets,
	each one of which is closed in $Y$.
	The empty set is not consider to be irreducible.
   \end{definition}
   
   As an exercise we can prove show that 
   \begin{proposition}
   	Any nonemepty open subset of an irreducible space is irreducible
	and dense. 
   \end{proposition}
   
   \begin{proof}
	Let $X$ be an irreducible topological space.
	Assume for contradiction that 
	there exists a non-empty $U$ 
	such that $\overline{U} \ne Y$,
	then we can write $X = (\overline{U})^c \overline{U}$, 
	contradicting the fact that $X$ is irreducible. 

	Assume for contradiction that 
	$U$ is a nonempty open set in $X$.
	We can write $U = A \cup B$ where 
	$A, B$ are closed in $U$. 
	Since $U$ is dense, 
	$\overline{U} = \overline{A \cup B} = \overline{A} \cup \overline{B} = X$.
	Since $X$ is irreducible, 
	$\overline{A} = X$ without loss of generality.
	Since $A$ is closed in $U$, 
	$\overline{A} = \overline{A} \cap U = X \cap U = U$ and $A = U$.
	Therefore, $U$ cannot be written 
	as union of proper nonempty closed subsets, 
	hence irreducible. 			
   \end{proof}
   
   \begin{example}
   	Consider the algebraic set $Z(xy) \subseteq \A_{\C}^n$ .
	The locus consists of the $x$-axis and the $y$-axis
	and can be expressed as 
	\[
	Z(xy) = Z(x) \cup Z(y),
	\]
	where $Z(x)$ and $Z(y)$ are both closed in $Y$. 
	Thus this algebraic set is not irreducible.. 
   \end{example}
   
   \begin{definition}[Affine Algebraic Variety]
   	An \textbf{affine algebraic variety} (or \textbf{affine variety})
	is an irreducible closed subset of $\A^n$ under Zariski topology.
	An open subset of an affine variety is \textbf{quasi-affine variety}.
   \end{definition}
   
   A natural questions to ask 
   when given a geometric or topological space is:
   ``what is a good compactification of the space?"
   For an affine space $\A^n$, 
   a natural compactification is the projective space $\P^n$.
   But why? 
   Here is a thought experiement that leads to this intuition.
   Consider two lines in $\A^2$ space,
   and calculate the tangent value of the angle $\theta$ at their intersection
   if they intersect.
   If the two lines entirely overlap, $\tan(\theta) = 0$;
   if the two lines intersect at finitely many points, $-\infty < \tan(\theta) < \infty$.
   But if the two lines are parallel to each other,
   $\tan(\theta) = \infty$, whose information is not in the affine space;
   we lose the limit of infinity in the affine space. 
   For each affine space, we ``add the points of infinity" at the end.
   
   \begin{definition}[Projective $n$-space]
   	We define \textbf{projective $n$-space} over $k$,
	denoted $\P_k^n$, or simply $\P^n$,
	to be the set of equivalence classes of $(n + 1)$-tuples
	$(a_0, \ldots, a_n)$ of elements of $k$, 
	not all zero,
	under the equivalence relation given by 
	$(a_0, \ldots, a_n) \sim (\lambda a_0, \ldots, \lambda a_n)$ 
	for all $\lambda \in k$, $\lambda \ne 0$. 
   \end{definition} 
   
   Similarly as in the affine $n$-space,
   we can have concepts of algebraic set, irreducible algebraic sets
   (algebraic varieties) and Zariski topology on the projective space.
   
   \begin{definition}[Projective Algebraic Variety]
   	A \textbf{projective algebraic variety} (or simply \textbf{projective variety}) is an irreducible algebraic set in $\P^n$, 
	with the induced topology.
	An open subset of a projective variety is a 
	\textbf{quasi-projective variety}.
	The dimension of a projective or quasi-projective variety 
	is its dimension as a topological space.
   \end{definition}
   
   Algebraic varieties and projective varieties are sometimes referred to as 
   algebraic curves or projective algebraic curves,
   which people use interchangeably. 