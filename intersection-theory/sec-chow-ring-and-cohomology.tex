\section{Chow Ring and Cohomology}
From previous chapter, 
we learned that cohomology of a variety (or scheme) $X$
gives algebraic information about the intersection of subvarieties in $X$.
From a category theoretical point of view,
the cohomology is a contravariant functor from the category of algebraic varieties (or schemes) to the category of graded rings. 
Here we have seen that Chow ring is the exact analogy of cohomology.
We know that the Chow group $A^k(X)$ is generated 
by all the subdivarieties in $X$ of dimension $k$ 
with the rational equivalence relations between the subvarieties.
In the case of cohomology, the equivalence is homological equivalence 
induced by the coboundary maps, as we discussed before. 

Furthermore, for nonsingular toric varieties that have structure of 
a smooth manifold in the analytic topology, 
the Chow ring coincides with the de Rham cohomology of the variety.
This condition is sufficient for us to eventually calculate the 
cohomology ring of the heavy/light Hassett spaces using Chow ring. 

In this manuscript, we use $A^\ast(X)$ for the Chow ring of $X$
(not to be confused with the Chow group of $X$, $A(X)$) 
and $H^\ast(X; \Z)$ for the cohomology of $X$ with integer coefficients. Since this notation overrides the dual notation,
we denote the dual of the Chow ring (the intersection ring) as $A_\ast(X)$ 
and denote the homology using $H_\ast(X; \Z)$.
Notice that our previous discussion in this section states that
the intersection ring coincides with the homology 
for nonsingular toric varieties that have structure of a smooth manifold
in the analytic topology:
\[
A_\ast(X) \cong H_\ast(X; \Z).
\]

\subsection{Chow ring of projective $n$ space via toric variety and its associated fan}
	In fact, since $\P^n$ can be seen as toric varieties,
	the Chow ring can be further computed using combinatorial structures
	provided by the associated fan of the embedded toric variety.
	
	\begin{theorem}
		For a nonsingular projective variety $X$ 
		and its associated fan $\Sigma$,
		$A^\ast(X) = \Z[D_1, \ldots, D_d]/I$,
		where the $I$ is the ideal generated by all the 
		\begin{enumerate}	
			\item[(1)]
			products of divisors whose associated primitive generators 
			do not form a cone in the fan;
			that is 
			\[
			D_{i_i} \times \cdots \times D_{i_k}
			\]
			where $v_{i_1}, \ldots, v_{i_k}$ do not form a cone 
			in the fan $\Sigma$. 
			
			\item[(2)]
			$\sum\limits_{i = 1}^{d} \inner{u, v_i} D_i$
			for all the basis element $u$ that spans 
			the whole ambient vector space. 
		\end{enumerate}
	\end{theorem}
	
Let us calculate some Chow ring in some familiar spaces to practice the notion we just learned. 
	\begin{example}[Chow ring of $\P^3$]
		Recall that the primitive generators are 
		\begin{align*}
		v_1 &= (-1, -1, -1), v_2 = (1, 0, 0) \\
		v_3 &= (0, 1, 0), v_4 = (0, 0, 1) \\
		\end{align*}
		and the maximal cones are spanned by the following 
		sets of primitive generators
		\begin{align*}
		S_1 = \{v_1, v_2, v_3\}, \\
		S_2 = \{v_1, v_2, v_4\}, \\
		S_3 = \{v_1, v_3, v_4\}, \\
		S_4 = \{v_2, v_3, v_4\}, \\
		\end{align*}
		since four of the primitive generators together 
		do not span any cone. 
		We can check that the lower dimensional cones 
		can be found by intersecting the higher dimensional cones,
		which satisfy the condition for a fan. 
		
		To construct calculate Chow ring, we following 
		the previous theorem to find the generators of the ideal.
		We know that they have two types:
		\begin{enumerate}
			\item[(1)] The first type is the product of 
				all the primitive generators that do not span a cone.
				Notice that the only set of generators that 
				do not span a cone is the set full set 
				$\{v_1, v_2, v_3, v_4\}$,
				which gives us 
				\[
				D_1 D_2 D_3 D_4 \in I.
				\]
			\item[(2)] 
				The ambient vector space can be seen as 
				spanned by $\calb = \{v_1, v_2, v_3\}$,
				and thus this is a basis.
				Now for each basis element $u \in \calb$,
				we set 
				\[
				\sum\limits_{i = 1}^{d} \inner{u, v_i} D_i = 0. 
				\] 
				Thus for the basis element $v_1$, we have
				\[
				\sum\limits_{i = 1}^d \inner{v_1, v_i} D_i 
				= 1 \cdot D_1 + 0 \cdot D_2 + 0 \cdot D_3 - 1 \cdot D_4 = D_1 - D_4 = 0,
				\] and thus $D_1 = D_4$. 
				For $v_2$, we have 
				\[
				\sum\limits_{i = 1}^d \inner{v_2, v_i} D_i 
				= 0 \cdot D_1 + 1 \cdot D_2 + 0 \cdot D_3 - 1 \cdot D_4 = D_1 - D_4 = 0,
				\] and thus $D_2 = D_4$. 
				For $v_3$, we have 
				\[
				\sum\limits_{i = 1}^d \inner{v_2, v_i} D_i 
				= 0 \cdot D_1 + 0 \cdot D_2 + 1 \cdot D_3 - 1 \cdot D_4 = D_3 - D_4 = 0,
				\] and thus $D_3 = D_4$. 
				Therefore, 
				we have obtained another equivalence relation 
				that we need to mod out 
				\[
				D_1 = D_4, D_2 = D_4, D_3 = D_4.
				\]
				The generator of the ideal 
				that we obtain from step (1)
				can be rewritten using this relation as 
				\[
				D_1 D_2 D_3 D_4 = D_4^4
				\]
				Since now the divisors are all the same,
				renaming $D_i = H$, 
				we obtain the Chow ring of $\P^3$
				\[
				A^\ast(X) = \Z[H]/\inner{H^4}
				\]
				which recovers the result from the theorem 
				in previous section 
				where the proof does not use the combinatorial 
				structure of the associated fan of the toric variety 
				but only the geometric information 
				of the intersection of the subvarieties. 	
		\end{enumerate}
	
	\end{example}	
	
	\begin{example}[Chow ring of $(\P^1)^3$]
		Recall that the primitive generators are 
		\begin{align*}
		v_1 &= (1, 0, 0), \\
		v_2 &= (-1, 0, 0), \\
		v_3 &= (0, 1, 0), \\
		v_4 &= (0, -1, 0), \\
		v_5 &= (0, 0, 1), \\
		v_6 &= (0, 0, -1),
		\end{align*}
		The maximal cones are again generated by 
		every three of those generators.
		But since there are linearly dependent generators,
		we can only form $8$ top-dimensional (dimension $3$) cones,
		spanned by the following spanning sets:
		\begin{align*}
		S_1 = \{v_1, v_3, v_5\}, \\
		S_2 = \{v_1, v_3, v_6\}, \\
		S_3 = \{v_1, v_4, v_5\}, \\
		S_4 = \{v_1, v_5, v_6\}, \\
		S_5 = \{v_2, v_3, v_5\}, \\
		S_6 = \{v_2, v_3, v_6\}, \\
		S_7 = \{v_3, v_4, v_5\}, \\
		S_8 = \{v_4, v_5, v_6\}, \\
		\end{align*}
		
		Readers might already realize that 
		generators that are opposite directions cannot generate 
		higher dimensional cones.
		Thus the following sets cannot generate non-trivial cones
		$\{D_1, D_2\}$, $\{D_3, D_4\}$, $\{D_5, D_6\}$
		and all the unions of the three sets. 
		Thus we have
		\[
		D_1D_2 = 0, D_3 D_4 = 0, D_5 D_6 = 0,
		\]
		for the first type relation that yield the generators of the ideal.
		Then since the ambient vector space is $3$-dimensional,
		we select a basis $\calb = \{v_1, v_3, v_5\}$
		and for each basis element we do the following:
		Now for each basis element $u \in \calb$,
		we set 
		\[
		\sum\limits_{i = 1}^{d} \inner{u, v_i} D_i = 0. 
		\] 

		 For the basis element $v_1$, we have
				\[
				\sum\limits_{i = 1}^d \inner{v_1, v_i} D_i 
				= D_1 -  D_2  = 0, 
				\] and thus $D_1 = D_2$. 
				For $v_3$, we have 
				\[
				\sum\limits_{i = 1}^d \inner{v_3, v_i} D_i 
				=  D_3 -  D_4  = 0, 
				\] and thus $D_3 = D_4$. 
				For $v_5$, we have 
				\[
				\sum\limits_{i = 1}^d \inner{v_2, v_i} D_i 
				= D_5 - D_6  = 0,
				\] and thus $D_5 = D_6$. 
		In summery, the relations we have is thus
		\[
		D_1^2 = D_3^2 = D_5^2 = 0.
		\]
		Thus writing $D_1 = X, D_3 = Y, D_3 = Z$,
		the Chow ring of $\P^3$ is thus $\Z[X, Y, Z]$.
	\end{example}
	
	We state the general formula for projective $n$-space. 
	\begin{theorem}
		The Chow ring of $\P^n$ is 
		\[
		A(\P^n) = \Z[\zeta]/(\zeta^{n+1})
		\],
		where we say that $\zeta \in A^1(\P^n)$ 
		is the rational equivalence class of a hyperplane;
		ore more generally, the class of a variety of codimension $k$ 
		and degree $d$ is $d \zeta^k$.
	\end{theorem}