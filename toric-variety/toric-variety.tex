\chapter{Toric Variety}
\label{chp:toric-variety}

\epigraph{
Toric varieties have provided a remarkably fertile testing ground for general theories.
}{\citet{Fulton1993}}

The theory of toric variety is one of the most important interplays 
between algebraic geometry and combinatorics. 
Since its introduction in 1970s, toric varieties have continued to provide 
quite special and powerful tools to view classical phenomena 
in algebraic geometry.
In particular, it stands in the center of many important ideas 
including intersection theory and Riemann-Roch problem.
Amongst all of its powers and applications, 
toric varieties stands out particularly for its strong connection with 
the study of compactification problems. 
This aspect is also what the idea of this paper treats on. 

Loosely put, toric variety is a variety $X$
that contains a torus $T$ as a dense open subset,
together with the action on the torus by itself extended to the whole variety.

In this chapter, we will see that toric varieties correspond to objects,
called polyhedral fans,
that bear a passing resemblance to simplicial complices in algebraic topology.
This association concretizes everything and bestows powerful computational and combinatorial tools.
In this paper particularly, 
these polyhedral fans correspond exactly to the tropical moduli spaces
and they provide descriptions of toric varieties 
where the original moduli spaces of curves are embedded in.
In other words, the tropicalizaton of the moduli spaces 
tells us where to find a compactification of the moduli spaces. 
Then, finding the cohomology ring is the story for next chapter.
For readers who know of toric variety from a scheme theory point of view,
we will not dive into details of that aspect of toric variety,
which we apologize in advance. 
For further readings, readers may refer to 
\emph{Introduction to Toric Variety} by \citet{Fulton1993} 
and Chapter 7 of \emph{Mirror Symmetry} by \citet{Hori2003},
whose treatment of this topic is followed here. 